\documentclass[../users_guide.tex]{subfiles}
 
\begin{document}

\section{Using the REST VOL connector within an HDF5 application}

This section outlines the unique aspects of writing, building and running
\acrshort{hdf5} applications with the \rvc{}.

\subsection{Building the HDF5 REST VOL connector}

TODO - refer to README.md

%The following is a quick set of instructions for building the \rvc{} connector.
%Note that these instructions are not comprehensive and may be subject to change
%in future releases; please refer to the \rvc{}'s
%\href{https://bitbucket.hdfgroup.org/projects/HDF5VOL/repos/rest/browse/README.md}{README}
%file for the most up to date instructions.

\subsection{Writing HDF5 REST VOL connector applications}

There are currently two main ways to tell an existing \acrshort{hdf5} application to use
the \rvc{}: either \textit{implicitly} by using environment
variables to tell the \acrshort{hdf5} library to load the connector as a dynamically loaded
plugin or \textit{explicitly} by making use of \acrshort{hdf5} property lists.

\subsubsection{With the REST VOL connector as a dynamically-loaded plugin}
\label{sec:dynamic_plugin}

\acrshort{hdf5} has the capability to dynamically load and use a \vc{} for running
applications with. In order to choose a particular \vc{} to use, two
initial steps must be taken. First, one must help \acrshort{hdf5} locate the \vc{}
by pointing to the directory which contains the built library. This can be
accomplished by setting the environment variable \texttt{HDF5\_PLUGIN\_PATH} to
this directory. Next, \acrshort{hdf5} needs to know the name of which library to use, which
is configured by setting the environment variable \texttt{HDF5\_VOL\_CONNECTOR}
to the name of the connector.

In order to use the \rvc{}, the aforementioned environment variables 
should be set as:

\begin{verbatim}
HDF5_PLUGIN_PATH=/rest/vol/installation/directory/lib
HDF5_VOL_CONNECTOR=REST
\end{verbatim}

Having completed this step, \acrshort{hdf5} will be setup to load the \rvc{}
and use it for running applications, including \acrshort{hdf5}'s own tests.
No additional modifications will need to be made to the existing \acrshort{hdf5} application.

\subsubsection{Without the REST VOL connector as a dynamically-loaded plugin}

If dynamic loading of the \rvc{} is not used, any \acrshort{hdf5} application
using the connector must:
\begin{enumerate}
 \item Include \texttt{rest\_vol\_public.h}, found in the \texttt{include}
directory of the \rvc{} installation directory.
 \item Link against \texttt{libhdf5\_vol\_rest.so} (or similar), found in
the \texttt{lib} directory of the \rvc{} installation directory. Note that dependencies
can alternatively be retrieved through CMake or pkg-config.
\end{enumerate}

In this particular case, an \acrshort{hdf5} \rvc{} application will also require three new
function calls in addition to those for an equivalent \acrshort{hdf5} application (see
Appendix~\ref{apdx:ref_manual} for more details):

\begin{itemize}
 \item \texttt{\hyperref[ref:h5rest_init]{H5rest\_init()}} --- Initializes the \rvc{}

    Called upon application startup, before any file is accessed.

 \item \texttt{\hyperref[ref:h5pset_fapl_rest_vol]{H5Pset\_fapl\_rest\_vol()}} --- Sets \rvc{} access on a File Access Property List.

    Called to prepare a FAPL to open a file through the \rvc{}. See \href{https://support.hdfgroup.org/HDF5/Tutor/property.html#fa}{HDF5 File Access Property Lists} for more information about File Access Property Lists.

 \item \texttt{\hyperref[ref:h5rest_term]{H5rest\_term()}} --- Cleanly shutdowns the \rvc{}

    Called on application shutdown, after all files have been closed.
\end{itemize}

\subsubsection{Skeleton Example}

Below is a no-op application that opens and closes a file using the \rvc{}.
For clarity, no error-checking is performed. Note that this example is
meant only for the case when the \rvc{} is not being dynamically loaded.

\begin{minted}[breaklines=true,fontsize=\small]{hdf5-c-lexer.py:HDF5CLexer -x}
#include "hdf5.h"
#include "rest_vol_public.h"

int main(void)
{
    hid_t file_id;
    hid_t fapl_id;

    /* Initialize REST VOL connector */
    H5rest_init();

    fapl_id = H5Pcreate(H5P_FILE_ACCESS);
    H5Pset_fapl_rest_vol(fapl_id); 

    file_id = H5Fopen("my_file.h5", H5F_ACC_RDWR, fapl_id);

    /* Operate on file */
    [...]

    H5Pclose(fapl_id);
    H5Fclose(file_id);

    /* Terminate the REST VOL connector. */
    H5rest_term();

    return 0;
}
\end{minted}

\subsection{Building HDF5 REST VOL connector applications}

Assuming an \acrshort{hdf5} application has been written following the instructions in the previous
section, the application should be built as normal for any other \acrshort{hdf5} application. However,
if the \rvc{} is not being dynamically loaded, the steps in the following section are required to
build the application.

\subsubsection{Without the REST VOL connector as a dynamically-loaded plugin}

To link in the required libraries, the compiler will likely require the
additional linker flags:

\begin{verbatim}
-lhdf5_vol_rest -lcurl -lyajl
\end{verbatim}

However, these flags may vary depending on platform, compiler and installation
location of the \rvc{}. It is highly recommended that compilation
of \acrshort{hdf5} \rvc{} applications be done using either the
\texttt{h5cc} script included with \acrshort{hdf5} distributions, or CMake/
pkg-config, as these will manage linking with the \acrshort{hdf5} library.
The above notice about additional library linking applies to usage of
\texttt{h5cc/h5pcc}. For example:
\begin{verbatim}
h5cc -lhdf5_vol_rest -lcurl -lyajl my_application.c -o my_application
\end{verbatim}

\subsection{Running HDF5 REST VOL connector applications}
\label{running_rest_vol_apps}

\subsubsection{Server Access}

Running applications that use the \rvc{} connector requires access to a running HDF Kita
server. Refer to \href{https://www.hdfgroup.org/hdf-kita}{HDF Kita} for more information on
the setup process for this.

\subsubsection{Connection Information}

For the \rvc{} to correctly interact with a running HDF Kita server instance, the connector
must be passed the base URL of the HDF Kita endpoint, as well as any authentication credential
information needed. This can be accomplished in one of the following ways:

\textbf{Environment Variables}

\begin{itemize}
  \item \texttt{HSDS\_USERNAME} - (optional) The username to use for authentication
  \item \texttt{HSDS\_PASSWORD} - (optional) The password to use for authentication
  \item \texttt{HSDS\_ENDPOINT} - The base URL of the HDF Kita instance (e.g. http://hsdshdflab.hdfgroup.org)
\end{itemize}

\textbf{Configuration File}

TODO

Note that there are cases where authentication may not be required, such as when simply
retrieving information from a publicly-accessible HDF5 dataset or similar. In these cases, it
is only necessary to supply the HDF Kita endpoint via the HSDS\_ENDPOINT environment variable
or the .hscfg configuration file.

\subsubsection{Example Applications}

Some of the example C applications which are included with \acrshort{hdf5} distributions have
been adapted to work with the \rvc{} and are included under the top-level
\href{https://bitbucket.hdfgroup.org/projects/HDF5VOL/repos/rest/browse/examples}{\texttt{examples}}
directory in the \rvc{} source root directory. The built example applications can be run from the
\texttt{bin} directory inside the build directory.

In addition to these examples, the \href{https://bitbucket.hdfgroup.org/projects/HDF5VOL/repos/rest/browse/test}{\texttt{test/vol-tests}}
directory contains several test files, each containing test functions that are examples of \acrshort{hdf5}
applications in miniature, focused on a particular behavior. These mini-application tests cover a moderate
amount of \acrshort{hdf5}'s public API functionality and should be a good indicator of whether the \rvc{}
is working correctly in conjunction with a running HDF Kita instance. Note that these tests currently rely
on \acrshort{hdf5}'s dynamically-loaded \vc{} capabilities in order to run with the \rvc{}.

\end{document}
