\documentclass[../users_guide.tex]{subfiles}
 
\begin{document}

\section{HDF5 API Support}

\subsection{Feature Specific Support}

% TODO: H5Pset_create_intermediate_group support

The following sections serve to illustrate the \rvc{}'s support for features in \acrshort{hdf5}, as well as to highlight any differences between the expected behavior of an \acrshort{hdf5} feature versus the actual behavior as implemented by the VOL connector.

\subsubsection{Attribute Features}

\begin{tabularx}{\linewidth}{| X | X | X | X | >{\RaggedRight}X |}
\hline
\rowcolor{lightgray!50}%
% Center just the header row
\multicolumn{3}{| c |}{\textbf{Feature}} & \multicolumn{1}{c |}{\textbf{Supported?}} & \multicolumn{1}{c |}{\textbf{Notes}} \\ \hline

\multirow[c]{3}{\linewidth}{Dataspace} & \multirow[c]{3}{\linewidth}{Dimensionality} & H5S\_NULL & Yes & \\ \cline{3-4}
& & H5S\_SCALAR & Yes & \\ \cline{3-4}
& & SIMPLE & Yes & \\ \cline{3-4} \hline

\multicolumn{2}{| l |}{\multirow[c]{6}{*}[-90pt]{Datatype}} & Atomic & Yes & Non-predefined integer and floating-point datatypes are currently not supported.\\ [30pt] \cline{3-4}
\multicolumn{2}{| l |}{} & Compound & Yes & \\ [30pt] \cline{3-4}
\multicolumn{2}{| l |}{} & Variable-length & No & \\ [30pt] \cline{3-4}
\multicolumn{2}{| l |}{} & Array & Yes & \\ [30pt] \cline{3-4}
\multicolumn{2}{| l |}{} & String & Yes & Only \texttt{H5T\_STR\_NULLPAD} is supported for string padding for fixed-length strings. Only \texttt{H5T\_STR\_NULLTERM} is supported for string padding for variable-length strings. \\ [30pt] \cline{3-4}
\multicolumn{2}{| l |}{} & Enum & Yes & \\ [30pt] \cline{3-4}
\multicolumn{2}{| l |}{} & Opaque & No & \\ [30pt] \cline{3-4}
\multicolumn{2}{| l |}{} & Bitfield & No & \\ [30pt] \cline{3-4}
\multicolumn{2}{| l |}{} & Time & No & \\ [30pt] \cline{3-4}
\multicolumn{2}{| l |}{} & Reference & No & \\ [30pt] \cline{3-4}
\hline

\multirow[c]{2}{\linewidth}{Properties} & \multirow[c]{2}{\linewidth}{Name Encoding} & ASCII & Yes & \\ \cline{3-5}
& & UTF-8 & No & \\ \cline{3-4} \hline

\end{tabularx}

\newpage

\subsubsection{Dataset Features}

\begin{tabularx}{\linewidth}{| X | X | X | X | >{\RaggedRight}X |}
\hline
\rowcolor{lightgray!50}%
% Center just the header row
\multicolumn{3}{| c |}{\textbf{Feature}} & \multicolumn{1}{c |}{\textbf{Supported?}} & \multicolumn{1}{c |}{\textbf{Notes}} \\ \hline

\multirow[c]{7}{\linewidth}[-45pt]{Dataspace} & \multirow[c]{3}{\linewidth}{Dimensionality} & H5S\_NULL & Yes & \\ \cline{3-4}
& & H5S\_SCALAR & Yes & \\ \cline{3-4}
& & SIMPLE & Yes & \\ \cline{3-4} \cline{2-5}
& \multirow[c]{4}{\linewidth}[-55pt]{Selection Type} & NONE & Yes & \\ [28pt] \cline{3-4}
& & H5S\_ALL & Yes & \\ [28pt] \cline{3-4}
& & Hyperslab Selection & Yes & Non-regular and non-contiguous hyperslabs are currently not supported.\\ [28pt] \cline{3-4}
& & Point Selection & Yes & \\ [28pt] \cline{3-4} \hline

\multicolumn{2}{| l |}{\multirow[c]{6}{*}[-90pt]{Datatype}} & Atomic & Yes & Non-predefined integer and floating-point datatypes are currently not supported.\\ [30pt] \cline{3-4}
\multicolumn{2}{| l |}{} & Compound & Yes & \\ [30pt] \cline{3-4}
\multicolumn{2}{| l |}{} & Variable-length & No & \\ [30pt] \cline{3-4}
\multicolumn{2}{| l |}{} & Array & Yes & \\ [30pt] \cline{3-4}
\multicolumn{2}{| l |}{} & String & Yes & Only \texttt{H5T\_STR\_NULLPAD} is supported for string padding for fixed-length strings. Only \texttt{H5T\_STR\_NULLTERM} is supported for string padding for variable-length strings. \\ [30pt] \cline{3-4}
\multicolumn{2}{| l |}{} & Enum & Yes & \\ [30pt] \cline{3-4}
\multicolumn{2}{| l |}{} & Opaque & No & \\ [30pt] \cline{3-4}
\multicolumn{2}{| l |}{} & Bitfield & No & \\ [30pt] \cline{3-4}
\multicolumn{2}{| l |}{} & Time & No & \\ [30pt] \cline{3-4}
\multicolumn{2}{| l |}{} & Reference & No & \\ [30pt] \cline{3-4}
\hline

\end{tabularx}

\newpage

\begin{tabularx}{\linewidth}{| X | X | X | X | >{\RaggedRight}X |}
\hline
\rowcolor{lightgray!50}%
% Center just the header row
\multicolumn{3}{| c |}{\textbf{Feature}} & \multicolumn{1}{c |}{\textbf{Supported?}} & \multicolumn{1}{c |}{\textbf{Notes}} \\ \hline

\multirow[c]{9}{\linewidth}[-80pt]{Properties} & \multirow[c]{5}{\linewidth}[-35pt]{Storage Properties (creation)} & Compact & No & Setting is ignored; stored as contiguous. \\ \cline{3-5}
& & External & No & Setting is ignored; stored as contiguous. \\ \cline{3-5}
& & Contiguous & Yes & \\ \cline{3-5}
& & Chunked & Yes & \\ \cline{3-5}
& & VDS & No & The VDS feature is not currently planned to be supported.\\ \cline{2-5}
& \multirow[c]{4}{\linewidth}[-45pt]{Other Properties (creation)} & Attribute Creation Order & Yes & \\ \cline{3-5}
& & Fill Value & No & \\ \cline{3-5}
& & Filters & No & HDF5 does not expose any public APIs for working with the filter pipeline; however, this feature may be supported in the future. \\ \cline{3-5}
& & Storage Allocation Time & N/A & \\ \hline

\end{tabularx}

\newpage

\begin{tabularx}{\linewidth}{| X | X | X | X | >{\RaggedRight}X |}
\hline
\rowcolor{lightgray!50}%
% Center just the header row
\multicolumn{3}{| c |}{\textbf{Feature}} & \multicolumn{1}{c |}{\textbf{Supported?}} & \multicolumn{1}{c |}{\textbf{Notes}} \\ \hline

\multirow[c]{4}{\linewidth}[-55pt]{Properties (cont.)} & \multirow[c]{3}{\linewidth}[-50pt]{Access Properties} & Chunk cache & No & HDF5 does not expose any public APIs for implementing a chunk cache for arbitrary VOL connectors; however, this feature may be supported in the future. \\ \cline{3-5}
& & VDS views and printf & No & The VDS feature is not currently planned to be supported.\\ \cline{3-5}
& & MPI-I/O Collective Metadata Ops & No & \\ \cline{2-5}
& \multirow[c]{1}{\linewidth}[-10pt]{Transfer Properties} & MPI-I/O Independent or Collective I/O mode & N/A & \\ \hline

\end{tabularx}

\subsubsection{File Features}

\begin{tabularx}{\linewidth}{| X | X | X | X | >{\RaggedRight}X |}
\hline
\rowcolor{lightgray!50}%
% Center just the header row
\multicolumn{3}{| c |}{\textbf{Feature}} & \multicolumn{1}{c |}{\textbf{Supported?}} & \multicolumn{1}{c |}{\textbf{Notes}} \\ \hline

\multicolumn{2}{| l |}{\multirow[c]{2}{*}{File creation flags}} & \multicolumn{1}{l |}{H5F\_ACC\_TRUNC} & Yes & \multirow[c]{4}{\linewidth}{The file creation flags behave as for native HDF5.} \\ \cline{3-4}
\multicolumn{2}{| l |}{} & \multicolumn{1}{l |}{H5F\_ACC\_EXCL} & No & \\ \cline{1-4}
\multicolumn{2}{| l |}{\multirow[c]{2}{*}{File opening flags}} & \multicolumn{1}{l |}{H5F\_ACC\_RDWR} & Yes & \\ \cline{3-4}
\multicolumn{2}{| l |}{} & \multicolumn{1}{l |}{H5F\_ACC\_RDONLY} & Yes & \\ \hline

\multirow[c]{8}{\linewidth}[-100pt]{Properties} & \multirow[c]{1}{\linewidth}[-70pt]{Creation Properties} & Attribute Creation Order & Yes & The rest of the file creation properties are related to the native HDF5-specific file format. \\ \cline{2-5}

& \multirow[c]{7}{\linewidth}{Access Properties (Drivers)} & SEC2 Driver & N/A & \multirow[c]{4}{\linewidth}{These drivers are applicable to native HDF5 only.} \\ \cline{3-4}
& & Family Driver & N/A & \\ \cline{3-4}
& & Split Driver & N/A & \\ \cline{3-4}
& & Multi Driver & N/A & \\ \cline{3-4}
& & Core Driver & N/A & \\ \cline{3-4}
& & Log Driver & N/A & \\ \cline{3-5}
& & MPI-I/O & No & This feature may be supported in the future. \\ \hline

\end{tabularx}

\newpage

\begin{tabularx}{\linewidth}{| X | X | X | X | >{\RaggedRight}X |}
\hline
\rowcolor{lightgray!50}%
% Center just the header row
\multicolumn{3}{| c |}{\textbf{Feature}} & \multicolumn{1}{c |}{\textbf{Supported?}} & \multicolumn{1}{c |}{\textbf{Notes}} \\ \hline

\multirow[c]{10}{\linewidth}[-100pt]{Properties (cont.)} & \multirow[c]{8}{\linewidth}[-60pt]{Access Properties (Other)} & MPI-I/O Collective Metadata Ops & No & \\ \cline{3-5}

& & User block & N/A & \\ \cline{3-5}
& & Chunk Cache & No & HDF5 does not expose any public APIs for implementing a chunk cache for arbitrary VOL connectors; however, this feature may be supported in the future. \\ \cline{3-5}
& & Object flushing callbacks & N/A & \\ \cline{3-5}
& & File closing degree & N/A & \\ \cline{3-5}
& & Evict on close & N/A & \\ \cline{3-5}
& & Sieve buffer size for partial I/O & No & HDF5 does not expose any public APIs for implementing a partial I/O sieve buffer for arbitrary VOL connectors; however, this feature may be supported in the future. \\ \cline{3-5}
& & File Image & N/A & \\ \hline

\end{tabularx}

\subsubsection{Group Features}

\begin{tabularx}{\linewidth}{| X | X | >{\RaggedRight}X | X | >{\RaggedRight}X |}
\hline
\rowcolor{lightgray!50}%
% Center just the header row
\multicolumn{3}{| c |}{\textbf{Feature}} & \multicolumn{1}{c |}{\textbf{Supported?}} & \multicolumn{1}{c |}{\textbf{Notes}} \\ \hline

\multirow[c]{4}{\linewidth}[-120pt]{Properties} & \multirow[c]{3}{\linewidth}[-100pt]{Creation Properties} & Link Creation Order & Yes & \\ \cline{3-5}
& & Attribute Creation Order & Yes & \\ \cline{3-5}
& & Other Properties & N/A & These properties are related to the native HDF5-specific file format. \\ \cline{2-5}
& \multirow[c]{1}{\linewidth}[-35pt]{Access Properties} & MPI-I/O Collective Metadata Ops & No & \\ \hline

\end{tabularx}

\newpage

\subsection{API Specific Support}

The following sections serve to illustrate the \rvc{}'s support for the \acrshort{hdf5} API, as well as to highlight any differences between the expected behavior of an \acrshort{hdf5} API call versus the actual behavior as implemented by the VOL connector. If a particular \acrshort{hdf5} API call does not appear among these tables, it is most likely a native \acrshort{hdf5}-specific API call which cannot be implemented by non-native \acrshort{hdf5} VOL connectors. These types of API calls are listed among the tables in Appendix~\ref{apdx:native_calls}.

\newpage

\subsubsection{H5A interface}

\begin{center}

\textbf{Supported API calls}
\vspace{.2in} \\

\begin{tabularx}{\linewidth}{| X | >{\RaggedRight}X |}
\hline
\rowcolor{lightgray!50}%
% Center just the header row
\multicolumn{1}{| c |}{\textbf{API call}} & \multicolumn{1}{c |}{\textbf{Notes}} \\ \hline

H5Acreate(1/2) & \\ \hline
H5Acreate\_by\_name & \\ \hline
H5Aopen(\_by\_name) & \\ \hline
H5Aopen\_name & Deprecated in favor of H5A\_open\_by\_name\\ \hline
H5Awrite & \\ \hline
H5Aread & \\ \hline
H5Aclose & \\ \hline
H5Aiterate(2) & \\ \hline
H5Aiterate\_by\_name & \\ \hline
H5Aexists(\_by\_name) & \\ \hline
H5Adelete(\_by\_name) & \\ \hline
H5Aget\_name & \\ \hline
\end{tabularx}

\begin{tabularx}{\linewidth}{| X | >{\RaggedRight}X |}
\hline
\rowcolor{lightgray!50}%
% Center just the header row
\multicolumn{1}{| c |}{\textbf{API call}} & \multicolumn{1}{c |}{\textbf{Notes}} \\ \hline

H5Aget\_space & \\ \hline
H5Aget\_type & \\ \hline
H5Aget\_info(\_by\_name) & Of the four fields in the \texttt{H5A\_info\_t} struct:
                                     \begin{itemize}
                                         \item \texttt{corder\_valid} is currently alwasy set to FALSE
                                         \item \texttt{corder} is currently always set to 0
                                         \item \texttt{cset} is currently always set to \texttt{H5T\_CSET\_ASCII}
                                         \item \texttt{data\_size} is currently always set to 0
                                     \end{itemize}\\ \hline
H5Aget\_create\_plist & \\ \hline

\end{tabularx}

\textbf{Currently unsupported API calls}
\vspace{.1in} \\

\begin{tabularx}{\linewidth}{| X | >{\RaggedRight}X |}
\hline
\rowcolor{lightgray!50}%
% Center just the header row
\multicolumn{1}{| c |}{\textbf{API call}} & \multicolumn{1}{c |}{\textbf{Notes}} \\ \hline

H5Aopen\_by\_idx & \\ \hline
H5Aopen\_idx & Deprecated in favor of H5Aopen\_by\_idx\\ \hline
H5Aget\_name\_by\_idx & \\ \hline
H5Aget\_info\_by\_idx & \\ \hline
H5Aget\_storage\_size & \\ \hline
H5Adelete\_by\_idx & \\ \hline
H5Arename(\_by\_name) & \\ \hline

\end{tabularx}

\end{center}

\newpage

\subsubsection{H5D interface}

\begin{center}

\textbf{Supported API calls}
\vspace{.2in} \\

\begin{tabularx}{\linewidth}{| X | >{\RaggedRight}X |}
\hline
\rowcolor{lightgray!50}%
% Center just the header row
\multicolumn{1}{| c |}{\textbf{API call}} & \multicolumn{1}{c |}{\textbf{Notes}} \\ \hline

H5Dcreate(1/2) & \\ \hline
H5Dcreate\_anon & \\ \hline
H5Dopen(1/2) & \\ \hline
H5Dwrite & \\ \hline
H5Dread & \\ \hline
H5Dclose & \\ \hline
H5Dget\_space & \\ \hline
H5Dget\_type & \\ \hline
H5Dget\_create\_plist & \\ \hline
H5Dget\_access\_plist & \\ \hline

\end{tabularx}

\textbf{Currently unsupported API calls}
\vspace{.2in} \\

\begin{tabularx}{\linewidth}{| X | >{\RaggedRight}X |}
\hline
\rowcolor{lightgray!50}%
% Center just the header row
\multicolumn{1}{| c |}{\textbf{API call}} & \multicolumn{1}{c |}{\textbf{Notes}} \\ \hline

H5Dget\_space\_status & \\ \hline
H5Dget\_storage\_size & \\ \hline
H5Dextend & \\ \hline
H5Dset\_extent & \\ \hline
H5Dflush & \\ \hline
H5Drefresh & \\ \hline

\end{tabularx}

\end{center}

\newpage

\subsubsection{H5F interface}

\begin{center}

\textbf{Supported API calls}
\vspace{.2in} \\

\begin{tabularx}{\linewidth}{| X | >{\RaggedRight}X |}
\hline
\rowcolor{lightgray!50}%
% Center just the header row
\multicolumn{1}{| c |}{\textbf{API call}} & \multicolumn{1}{c |}{\textbf{Notes}} \\ \hline

H5Fcreate & \\ \hline
H5Fopen & \\ \hline
H5Freopen & \\ \hline
H5Fget\_create\_plist & \\ \hline
H5Fget\_access\_plist & \\ \hline
H5Fget\_intent & \\ \hline
H5Fget\_name & \\ \hline
H5Fclose & \\ \hline

\end{tabularx}

\textbf{Currently unsupported API calls}
\vspace{.2in} \\

\begin{tabularx}{\linewidth}{| X | >{\RaggedRight}X |}
\hline
\rowcolor{lightgray!50}%
% Center just the header row
\multicolumn{1}{| c |}{\textbf{API call}} & \multicolumn{1}{c |}{\textbf{Notes}} \\ \hline

H5Fis\_accessible & \\ \hline
H5Fflush & \\ \hline
H5Fmount & \\ \hline
H5Funmount & \\ \hline
H5Fdelete & \\ \hline
H5Fget\_obj\_count & \\ \hline
H5Fget\_obj\_ids & \\ \hline

\end{tabularx}

\end{center}

\newpage

\subsubsection{H5G interface}

\begin{center}

\textbf{Supported API calls}
\vspace{.2in} \\

\begin{tabularx}{\linewidth}{| X | >{\RaggedRight}X |}
\hline
\rowcolor{lightgray!50}%
% Center just the header row
\multicolumn{1}{| c |}{\textbf{API call}} & \multicolumn{1}{c |}{\textbf{Notes}} \\ \hline

H5Gcreate(1/2) & \\ \hline
H5Gcreate\_anon & \\ \hline
H5Gopen(1/2) & \\ \hline
H5Gclose & \\ \hline
H5Gunlink & \\ \hline
H5Gget\_create\_plist & \\ \hline
H5Gget\_info(\_by\_name) & Of the four fields in the \texttt{H5G\_info\_t} struct:
                                     \begin{itemize}
                                         \item \texttt{storage\_type} is currently always set to \texttt{H5G\_STORAGE\_TYPE\_SYMBOL\_TABLE}
                                         \item \texttt{nlinks} is set appropriately
                                         \item \texttt{max\_corder} is currently always set to 0
                                         \item \texttt{mounted} is currently always set to \texttt{FALSE}
                                     \end{itemize}\\ \hline
H5Gget\_linkval & \\ \hline
H5Gget\_num\_objs & \\ \hline
H5Glink(2) & Currently only hard and soft link creation are supported.\\ \hline
H5Gmove(2) & Refer to Notes for \texttt{H5Lmove}\\ \hline

\end{tabularx}

\textbf{Currently unsupported API calls}
\vspace{.2in} \\

\begin{tabularx}{\linewidth}{| X | >{\RaggedRight}X |}
\hline
\rowcolor{lightgray!50}%
% Center just the header row
\multicolumn{1}{| c |}{\textbf{API call}} & \multicolumn{1}{c |}{\textbf{Notes}} \\ \hline

H5Gget\_info\_by\_idx & \\ \hline
H5Gget\_objname\_by\_idx & \\ \hline
H5Gflush & \\ \hline
H5Grefresh & \\ \hline

\end{tabularx}

\end{center}

\newpage

\subsubsection{H5L interface}

\begin{center}

\textbf{Supported API calls}
\vspace{.2in} \\

\begin{tabularx}{\linewidth}{| X | >{\RaggedRight}X |}
\hline
\rowcolor{lightgray!50}%
% Center just the header row
\multicolumn{1}{| c |}{\textbf{API call}} & \multicolumn{1}{c |}{\textbf{Notes}} \\ \hline

H5Lcreate\_hard & Reference count tracking is not currently implemented, so objects will not be removed when the last hard link pointing to them is removed\\ \hline
H5Lcreate\_soft & \\ \hline
H5Lexists & \\ \hline
H5Literate(\_by\_name) & \\ \hline
H5Lvisit(\_by\_name) & \\ \hline
H5Ldelete & Reference count tracking is not currently implemented, so objects will not be removed when the last hard link pointing to them is removed\\ \hline

\end{tabularx}

\begin{tabularx}{\linewidth}{| X | >{\RaggedRight}X |}
\hline
\rowcolor{lightgray!50}%
% Center just the header row
\multicolumn{1}{| c |}{\textbf{API call}} & \multicolumn{1}{c |}{\textbf{Notes}} \\ \hline

H5Lget\_info & Of the five fields in the \texttt{H5L\_info\_t} struct:
                                     \begin{itemize}
                                         \item \texttt{type} is set appropriately
                                         \item \texttt{corder\_valid} is currently always set to FALSE
                                         \item \texttt{corder} is currently always set to 0
                                         \item \texttt{cset} is currently always set to \texttt{H5T\_CSET\_ASCII}
                                         \item \texttt{u} has member \texttt{address} or \texttt{val\_size} set appropriately based on whether the link is a hard link or not
                                     \end{itemize}\\ \hline
H5Lget\_val & \\ \hline

\end{tabularx}

\newpage

\textbf{Currently unsupported API calls}
\vspace{.2in} \\

\begin{tabularx}{\linewidth}{| X | >{\RaggedRight}X |}
\hline
\rowcolor{lightgray!50}%
% Center just the header row
\multicolumn{1}{| c |}{\textbf{API call}} & \multicolumn{1}{c |}{\textbf{Notes}} \\ \hline

H5Lcreate\_external & \\ \hline
H5Lcreate\_ud & \\ \hline
H5Lget\_info\_by\_idx & \\ \hline
H5Lget\_val\_by\_idx & \\ \hline
H5Lget\_name\_by\_idx & \\ \hline
H5Ldelete\_by\_idx & \\ \hline
H5Lcopy & \\ \hline
H5Lmove & \\ \hline

\end{tabularx}

\end{center}

\newpage

\subsubsection{H5O interface}

\begin{center}

\textbf{Supported API calls}
\vspace{.2in} \\

\begin{tabularx}{\linewidth}{| X | >{\RaggedRight}X |}
\hline
\rowcolor{lightgray!50}%
% Center just the header row
\multicolumn{1}{| c |}{\textbf{API call}} & \multicolumn{1}{c |}{\textbf{Notes}} \\ \hline

H5Oopen & \\ \hline
H5Oclose & \\ \hline
H5Olink & \\ \hline

\end{tabularx}

\newpage

\textbf{Currently unsupported API calls}
\vspace{.2in} \\

\begin{tabularx}{\linewidth}{| X | >{\RaggedRight}X |}
\hline
\rowcolor{lightgray!50}%
% Center just the header row
\multicolumn{1}{| c |}{\textbf{API call}} & \multicolumn{1}{c |}{\textbf{Notes}} \\ \hline

H5Oopen\_by\_token & \\ \hline
H5Oopen\_by\_idx & \\ \hline
H5Oget\_info(\_by\_name/\_by\_idx) & \\ \hline
H5Oincr\_refcount & \\ \hline
H5Odecr\_refcount & \\ \hline
H5Oexists\_by\_name & \\ \hline
H5Ovisit(1/2) & \\ \hline
H5Ovisit\_by\_name(1/2) & \\ \hline
H5Ocopy & \\ \hline
H5Oflush & \\ \hline
H5Orefresh & \\ \hline

\end{tabularx}

\end{center}

\newpage

\subsubsection{H5R interface}

\begin{center}

\textbf{Supported API calls}
\vspace{.2in} \\

\begin{tabularx}{\linewidth}{| X | >{\RaggedRight}X |}
\hline
\rowcolor{lightgray!50}%
% Center just the header row
\multicolumn{1}{| c |}{\textbf{API call}} & \multicolumn{1}{c |}{\textbf{Notes}} \\ \hline

&\footnotemark \\ \hline

\end{tabularx}

\footnotetext{The REST VOL connector has not yet been updated to use HDF5's new reference API.}

\textbf{Currently unsupported API calls}
\vspace{.2in} \\

\begin{tabularx}{\linewidth}{| X | >{\RaggedRight}X |}
\hline
\rowcolor{lightgray!50}%
% Center just the header row
\multicolumn{1}{| c |}{\textbf{API call}} & \multicolumn{1}{c |}{\textbf{Notes}} \\ \hline

H5Rcreate\_object & \\ \hline
H5Rcreate\_region & \\ \hline
H5Rcreate\_attr & \\ \hline
H5Ropen\_object & \\ \hline
H5Ropen\_region & \\ \hline
H5Ropen\_attr & \\ \hline
H5Rget\_obj\_type3 & \\ \hline
H5Rget\_file\_name & \\ \hline
H5Rget\_obj\_name & \\ \hline

\end{tabularx}

\end{center}

\newpage

\subsubsection{H5T interface}

\begin{center}

\textbf{Supported API calls}
\vspace{.2in} \\

\begin{tabularx}{\linewidth}{| X | >{\RaggedRight}X |}
\hline
\rowcolor{lightgray!50}%
% Center just the header row
\multicolumn{1}{| c |}{\textbf{API call}} & \multicolumn{1}{c |}{\textbf{Notes}} \\ \hline

H5Tcommit(1/2) & \\ \hline
H5Tcommit\_anon & \\ \hline
H5Topen(1/2) & \\ \hline
H5Tclose & \\ \hline
H5Tget\_create\_plist & \\ \hline

\end{tabularx}

\textbf{Currently unsupported API calls}
\vspace{.2in} \\

\begin{tabularx}{\linewidth}{| X | >{\RaggedRight}X |}
\hline
\rowcolor{lightgray!50}%
% Center just the header row
\multicolumn{1}{| c |}{\textbf{API call}} & \multicolumn{1}{c |}{\textbf{Notes}} \\ \hline

H5Tflush & \\ \hline
H5Trefresh & \\ \hline

\end{tabularx}

\end{center}

\newpage

\subsection{Known Limitations}

The following outlines the known current limitations of the \rvc{}.

\begin{itemize}
  \item Trying to open an object in a file by using a pathname where any component of the path, except for the last component, is a soft link is currently not supported. For example, trying to open a dataset by the pathname `/group/subgroup/soft\_link\_to\_dataset` should work. However, trying to open a dataset by the pathname `/group/soft\_link\_to\_group/soft\_link\_to\_dataset` will generally fail.
  \item Using a trailing '/' character on path names will currently cause problems with the connector and result in incorrect behavior.
  \item The use of HDF5 point selections for dataset writes will generally incur an additional memory overhead of approximately 4/3 the size of the original buffer used for the \texttt{H5Dwrite} call. This is due to needing a temporary copy of the buffer which is base64-encoded for the server transfer.
\end{itemize}

\end{document}
